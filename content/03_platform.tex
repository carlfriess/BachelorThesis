%%%%%%%%%%%%%%%%%%%%%%%%%%%%%%%%%%%%%%%%%%%%%%%%%%%%%%%%%%%%%%%%%%%%%%%
%%%%%%%%%%%%%%%%%%%%%%%%%%%%%%%%%%%%%%%%%%%%%%%%%%%%%%%%%%%%%%%%%%%%%%%
%%%%%                                                                 %
%%%%%     <file_name>.tex                                             %
%%%%%                                                                 %
%%%%% Author:      <author>                                           %
%%%%% Created:     <date>                                             %
%%%%% Description: <description>                                      %
%%%%%                                                                 %
%%%%%%%%%%%%%%%%%%%%%%%%%%%%%%%%%%%%%%%%%%%%%%%%%%%%%%%%%%%%%%%%%%%%%%%
%%%%%%%%%%%%%%%%%%%%%%%%%%%%%%%%%%%%%%%%%%%%%%%%%%%%%%%%%%%%%%%%%%%%%%%


\chapter{Platform}

The hardware platform used for this project is a combination of a custom designed PCB and a Texas Instruments Launchpad (LAUNCHXL-F28379D) featuring the TMS320F28379D micro-controller. The Launchpad provides a sold basis which includes all the components necessary to run the micro-controller. It then plugs into the custom PCB which accommodates all the necessary external components and incorporates connectors for all sensors and actuators.

\todo{References to MCU/Launchpad}
\todo{Picture(s) of PCB + Launchpad}

\section{Micro-controller}

The Texas Instruments TMS320F28379D was chosen as it provides high performance in terms of processing with two CPU cores and two Control Law Accelerators. In addition, it incorporates a wide range of versatile peripherals covering most types of interfaces used in the system. Furthermore, this processor is a dual-core version of the single core TMS320F28377S which proved to work very well in Escher.

\section{Ethernet Controller}

To establish a network connection on the test track an Ethernet connection was required. Since the micro-controller is not equipped with an Ethernet interface it needed to be included externally.

After considering several options we decided on the WIZnet W5500 Ethernet Controller. Beyond providing Ethernet support it also incorporates hardware implementations of ICMP, ARP, IPv4, TCP, UDP and other protocols. This is advantageous as it offloads the computation necessary to run the network stack from the main processor, providing better performance. Furthermore, the chip is widely used and therefore has good community support.

The micro-controller communicates with the Ethernet Controller over SPI but the W5500 also provides an interrupt line which can be configured to provide interrupts on events such as incoming packages.

\section{SD-Card}

In order to log telemetry data a form of non-volatile memory was needed. The data must be easily accessible and quickly retrievable. The obvious and most suitable choice is an SD-card, as it can be integrated into the SPI bus and provides large amounts of storage. At the same time it can be easily plugged into a laptop in order to retrieve the data in the field.

\section{External Analog-To-Digital Converter (ADC)}

The pressure sensors on the pod produce analog signals that need to be converted. Additionally, the pod incorporates a set of low-voltage batteries and it is necessary to monitor their voltage and the current consumption from them. Although the micro-controller features a built-in 12-bit ADC which could accomplish this task, we wanted to achieve higher precision using an external 24-bit ADC (ADS124S08). Communication with the ADC also runs over an SPI bus. However, the external ADC uses a different SPI mode than the Ethernet Controller and SD-card. Thus a separate SPI bus is required.

\section{RS485 Bus}

An objective in the design of this system, was to use as many digital sensors as possible. This was possible for both types of laser distance sensors used on the pod. Both support the RS485 serial bus. Using an appropriate RS485 transceiver, the Serial Communication Interface (SCI) of the micro-controller can be utilized almost natively to communicate with multiple sensors on a single RS485 bus. Unfortunately the two types of sensors use different bus settings and thus it was simpler to separate them into two buses with two sensors each.

Using the RS485 bus standard means less analog signals that are more prone to interference. It also allows for higher precision, as there are no precision losses during to conversions.

\section{CAN Bus}

The motor-controllers (inverters) used on the pod are designed for automotive applications, while the Battery Management Systems (BMS) are designed for aerospace applications. Therefore, both systems are designed to use the CAN bus for control and telemetry. The micro-controller incorporates a CAN bus and transceiver on the Launchpad makes communication with these devices possible without any additional hardware.

The main advantages of the CAN bus in this application are built-in bus arbitration and automatic retransmission. This allows devices on the CAN bus to transmit telemetry data asynchronously without the possibility of data loss.

%%%%%%%%%%%%%%%%%%%%%%%%%%%%%%%%%%%%%%%%%%%%%%%%%%%%%%%%%%%%%%%%%%%%%%%
%%%%%%%%%%%%%%%%%%%%%%%%%%%%%%%%%%%%%%%%%%%%%%%%%%%%%%%%%%%%%%%%%%%%%%%
%%%%%                                                                 %
%%%%%     <file_name>.tex                                             %
%%%%%                                                                 %
%%%%% Author:      <author>                                           %
%%%%% Created:     <date>                                             %
%%%%% Description: <description>                                      %
%%%%%                                                                 %
%%%%%%%%%%%%%%%%%%%%%%%%%%%%%%%%%%%%%%%%%%%%%%%%%%%%%%%%%%%%%%%%%%%%%%%
%%%%%%%%%%%%%%%%%%%%%%%%%%%%%%%%%%%%%%%%%%%%%%%%%%%%%%%%%%%%%%%%%%%%%%%

\chapter*{Abstract}

Hyperloop is a high-speed transportation concept, where pressurized capsules (“pods”) travel through vacuum tubes. The control system developed in this thesis is designed to control a second-generation prototype Hyperloop pod competing in a student competition hosted by SpaceX in Los Angeles, California. The main design objectives were correctness and reliability while improving performance in comparison to the control system of the previous pod. 

Although no real-time operating system was used, the system was designed to be fundamentally asynchronous and runs across two CPU cores and a Control Law Accelerator. It transmits telemetry data to a control panel over a network and logs all incoming sensor data, as well as derived data. Robust navigation and control algorithms guarantee safe operation of the vehicle and correct responses during failures.

The control system was deployed and tested in the field before and during the competition and performed perfectly. Although a manufacturing fault in the propulsion system prevented the team from advancing to the finals of the competition, the pod's design tackles many technical issues relating to high-speed transportation in a low-pressure environment.
